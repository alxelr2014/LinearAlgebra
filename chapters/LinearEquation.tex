\chapter{Linear Equation}
\thispagestyle{headings}
\section{Fields}
The set \(\Field\) together with two operation \(+\), addition, and \(\cdot\), multiplication, that satisfy the follwings is called a \textbf{field}. For all \(x,y,z \in \Field\)
\begin{enumerate}
    \item Addition and multiplication are \textit{commutative}
          \[ x + y = y + x \qquad \qquad x \cdot y = y \cdot x\]
    \item Addition and multiplication are \textit{associative}
          \[x + (y + z) = (x + y) + z \qquad \qquad x \cdot (y \cdot z) = (x \cdot y) \cdot z\]
    \item Multiplication distributes over addition
          \[x \cdot (y + z)  = x \cdot y +  x \cdot z\]
    \item There exists an element \(0\), zero, in \(\Field\) such that \(x + 0 = x\).
    \item There exists an element \(1\) , one, in \(\Field\) such that \(x \cdot 1 = x\).
    \item For each element \(x \in \Field\) there corresponds a unique element \( y \in \Field\) such that \(x + y  = 0\). \(y\) is commonly denoted as \(-x\).
    \item For each non-zero element \(x \in \Field\) there corresponds a unique element \( y \in \Field\) such that \(x \cdot y  = 1\). \(y\) is commonly denoted as \(x^{-1}\) or \(\frac{1}{x}\).
    \item \(\Field\) is closed under addition and multiplication.
          \[x + y \in \Field \qquad \qquad x \cdot y \in \Field\]
\end{enumerate}

\begin{definition} [Characteristics]
    Let \(n\) be the least number such that
    \begin{equation}
        \underbrace{1 + 1 + \dots 1}_{n} = 0
    \end{equation}
    then \(n\) is the \textbf{characteristics} of \(\Field\). If for a field there exists no such \(n\), then its characteristics is \(0\).
\end{definition}

\begin{theorem}
    If \(\Field\) is a finite field, then the number of elements of \(\Field\) must be in form of \(p^k\) where \(p\) isa prime number and \(k \in \Naturals\). Also fro every number in such form there exists a unique \(\Field\) with \(p^k\) elements.
\end{theorem}

If \(\Field\) is a field then the set of all polynomials with the coefficients in \(\Field\) is denoted by \(\squareFunc{\Field}{x}\), that is
\begin{equation*}
    \squareFunc{\Field}{x} = \left\{ \sum_{i = 0}^{n} a_i x^i \; \middle| \; a_i \in \Field \: \forall i,\ n \in \Naturals \right\}
\end{equation*}

Cleary \(\squareFunc{\Field}{x}\) does not have a multiplicative inverse for some of its non-zero elements. Define \(\func{\Field}{x}\) as follow
\begin{equation*}
    \func{\Field}{x} = \left\{ \frac{\func{f}{x}}{\func{g}{x}} \; \middle| \; \func{f}{x}, \func{g}{x} \in \squareFunc{\Field}{x}, \  \func{g}{x} \neq 0 \right\}
\end{equation*}

which is a field. Also, note that \(\Field \subset \squareFunc{\Field}{x} \subset \func{\Field}{x}\).
\section{Matrices}
Let us denote the set of all metrices of size \(m \times n\) with elements in \(\Field\) by \( \Matrices{m}{n} \) and if \(m = n\) then it is equivalently denoted as \(\SquareMatrices{n}\).

Matrix \(A \in \SquareMatrices{n}\) is said to be \textbf{invertiable} or \textbf{singular} if there exists a matrix \(B \in \SquareMatrices{n}\) such that \(AB = BA = \IdentityMatrix_n\).

Consider the following system of linear equations:
\begin{equation}
    \left\{
    \begin{alignedat}{3} \label{eq:LinearEquation}
        % R & L   &  R & L   &  R & L 
        a_{11}x_1 & +{} &  a_{12}x_2 & +{} & a_{1n}x_n & = y_1 \\
        a_{21}x_1 & +{} &  a_{22}x_2 & +{} & a_{2n}x_n & = y_2 \\
        & \vdots\\
        a_{m1}x_1 & +{} &  a_{n2}x_2 & +{} & a_{mn}x_n & = y_m
    \end{alignedat}
    \right.
\end{equation}
with all \(a_{ij} \in \Field\). Then if \(c_k \in F,\; k = 1, \dots, n\):
\begin{equation*}
    (c_1 a_{11} + c_2 a_{21} + \dots c_m a_{m1})x_1 + \dots + (c_1 a_{1n} + c_2 a_{2n} + \dots c_m a_{mn})x_n = c_1 y_1 + \dots + c_m y_m
\end{equation*}
is \textbf{linear combination} of the \Cref{eq:LinearEquation}.

\begin{definition}[Equaivalent Systems]
    Two systems are considered equivalent if each equation in one system is a linear combination of the other system.
\end{definition}

\begin{proposition}
    Equivalent systems of linear equations have exactly the same solution.
\end{proposition}

The linear system, \Cref{eq:LinearEquation}, can be represent in form of matrices \(AX = Y\) where \linebreak\({A = \begin{bmatrix}
            a_{11} & \dots  & a_{1n} \\
            \vdots & \ddots &        \\
            a_{n1} & \dots  & a_{mn}
        \end{bmatrix} \in \Matrices{m}{n}}\)
and \(X = \begin{bmatrix}
    x_1    \\
    \vdots \\
    x_n
\end{bmatrix} , Y = \begin{bmatrix}
    y_1    \\
    \vdots \\
    y_m
\end{bmatrix}\)
then the solutions of \Cref{eq:LinearEquation} are exactly the same as \(AX = Y\).

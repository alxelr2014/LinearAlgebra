\chapter{Linear Equation}
\thispagestyle{headings}
\section{Fields}
The set \(\Field\) together with two operation \(+\), addition, and \(\cdot\), multiplication, that satisfy the follwings is called a \textbf{field}. For all \(x,y,z \in \Field\)
\begin{enumerate}
    \item Addition and multiplication are \textit{commutative}
          \[ x + y = y + x \qquad \qquad x \cdot y = y \cdot x\]
    \item Addition and multiplication are \textit{associative}
          \[x + (y + z) = (x + y) + z \qquad \qquad x \cdot (y \cdot z) = (x \cdot y) \cdot z\]
    \item Multiplication distributes over addition
          \[x \cdot (y + z)  = x \cdot y +  x \cdot z\]
    \item There exists an element \(0\), zero, in \(\Field\) such that \(x + 0 = x\).
    \item There exists an element \(1\) , one, in \(\Field\) such that \(x \cdot 1 = x\).
    \item For each element \(x \in \Field\) there corresponds a unique element \( y \in \Field\) such that \(x + y  = 0\). \(y\) is commonly denoted as \(-x\).
    \item For each non-zero element \(x \in \Field\) there corresponds a unique element \( y \in \Field\) such that \(x \cdot y  = 1\). \(y\) is commonly denoted as \(x^{-1}\) or \(\frac{1}{x}\).
    \item \(\Field\) is closed under addition and multiplication.
          \[x + y \in \Field \qquad \qquad x \cdot y \in \Field\]
\end{enumerate}

\begin{definition} [Characteristics]
    Let \(n\) be the least number such that
    \begin{equation}
        \underbrace{1 + 1 + \dots 1}{n} = 0
    \end{equation}
    then \(n\) is the \textbf{characteristics} of \(\Field\). If for a field there exists no such \(n\), then its characteristics is \(0\).
\end{definition}

\begin{theorem}
    If \(\Field\) is a finite field, then the number of elements of \(\Field\) must be in form of \(p^k\) where \(p\) isa prime number and \(k \in \Naturals\). Also fro every number in such form there exists a unique \(\Field\) with \(p^k\) elements.
\end{theorem}

If \(\Field\) is a field then the set of all polynomials with the coefficients in \(\Field\) is denoted by \(\squareFunc{\Field}{x}\), that is
\begin{equation*}
    \squareFunc{\Field}{x} = \left\{ \sum_{i = 0}^{n} a_i x^i \; \middle| \; a_i \in \Field \: \forall i,\ n \in \Naturals \right\}
\end{equation*}

Cleary \(\squareFunc{\Field}{x}\) does not have a multiplicative inverse for some of its non-zero elements. Define \(\func{\Field}{x}\) as follow
\begin{equation*}
    \func{\Field}{x} = \left\{ \frac{\func{f}{x}}{\func{g}{x}} \; \middle| \; \func{f}{x}, \func{g}{x} \in \squareFunc{\Field}{x}, \  \func{g}{x} \neq 0 \right\}
\end{equation*}

which is a field. Also, note that \(\Field \subset \squareFunc{\Field}{x} \subset \func{\Field}{x}\).
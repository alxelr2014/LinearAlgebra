\chapter{Vector Spaces}
\thispagestyle{headings}

A \textbf{vector space}, also called \textbf{linear space}, consist of the followings:
\begin{enumerate}
    \item a field \(\Field\) of scalars.
    \item a set \(V\) of vectors.
    \item a vector addition \(+\), with the following properties:
          \begin{enumerate}
              \item \(V\) is closed under addition.
              \item addition is commutative.
              \item addition is associative.
              \item addition has a unique identity element \(0\).
              \item for each vector \(\alpha \in V\), \(\exists \beta \in V \; \suchThat \; a +b = 0\).
          \end{enumerate}
    \item a scalar multiplication with the following properties:
          \begin{enumerate}
              \item \(V\) is closed under scalar multiplication.
              \item \((c_1c_2)\alpha = c_1(c_2 \alpha)\).
              \item \(c(\alpha + \beta) = c \alpha + c \beta\).
              \item \((c_1 + c_2) \alpha = c_1 \alpha + c_2 \alpha\)
              \item scalar multiplication has a unique identity element \(1\).
          \end{enumerate}
\end{enumerate}

\begin{example}
    \(V = \Reals\) and \(\Field = \Reals\) is a vector space. Furthermore, \(V = \Reals^n\) over \(\Field = \Reals\) is a vector space with the scalar multiplication \(c (x_1,x_2,\dots,x_n) = (cx_1,cx_2,\dots ,cx_n)\).
\end{example}

\begin{example}
    \(V = \Reals\) and \(\Field = \Integers\) is not a vector space as \(\Integers\) is not a field.
\end{example}

\begin{definition}
    A vector \(\beta \in V\) is said to be a \textbf{linear combination} of the vectors \(\alpha_1, \alpha_2, \dots, \alpha_n\) if there exists scalars \(c_1, c_2, \dots, c_n \in \Field\) such that:
    \begin{equation*}
        \beta = c_1 \alpha_1 + c_2 \alpha_2 + \dots + c_n \alpha_n = \sum_{i = 1}^{n} c_i \alpha_i
    \end{equation*}
\end{definition}

\section{Subspaces}
Let \(V\) be a vector space over field \(\Field\). A susbspace of \(V\) is a subset \(W\) of \(V\) which is itself a vector space over \(\Field\) with the operations of vector addition and scalar multiplication on \(V\).

\begin{theorem}
    A non-empty subset \(W\) of \(V\) is a subspace of \(V\) if and only if for each pair of vectors \(\alpha, \beta \in W\) and each scalar \(c \in \Field\) the vector \(c\alpha + \beta \in W\).
\end{theorem}

\begin{proof}
    Necessity: Suppose \(W\) is a non-empty subset of \(V\) with the above property. Since \(W\) is not empty then there exists a vector \(\alpha \in W\) and therefore, \((-1) \alpha + \alpha = 0 \in W\). For each \(c \in \Field\), \(c\alpha + 0 = c\alpha \in W\). Finally, if \(\beta \in W\) as well then \(1 \alpha + \beta \in W\). Therefore, \(W\) satisfies all the conditions and is a linear subspace of \(V\).

    Sufficiency: If \(W\) is a subspace of \(V\) and \(\alpha, \beta \in W\) with \(c \in \Field\) then \(c\alpha + \beta \in W\).
\end{proof}

\begin{corollary} \label{th:IntersectionSubspace}
    Let \(V\) be a vector space over \(\Field\). The intersection of any collection subspaces of \(V\) is a subspace of \(V\).
\end{corollary}

\begin{theorem}
    Let \(V\) be a vector space and \(W_1\) and \(W_2\) be two subspaces of \(V\) such that \(W_1 \cap W_2\) is a subspace of \(V\). Then \(W_1 \subset W_2 \) or \(W_2 \subset W_1\).
\end{theorem}

\begin{proof}
    For the sake of contradiction assume neither \(W_1 \subset W_2\) nor \(W_2 \subset W_1\). Then, there are vectors \(\alpha_1\) and \(\alpha_2\) such that \(\alpha_1 \in W_1\) but \(\alpha_2 \notin W_2\) and similarly \(\alpha_2 \in W_2\) but \(\alpha_2 \notin W_1\). Since \(\alpha_1 + \alpha_2 \in W_1 \cup W_2\), then \(\alpha_1 + \alpha_2 \in W_1\) or \(\alpha_1 + \alpha_2 \in W_2\) which is a contradiction.
\end{proof}

\begin{theorem}
    Let \(V\) be a vector space over the inifinite field \(\Field\). If \(W_1,W_2,\dots, W_n\) are subspaces of \(V\) and \(V \subset \cup W_i\), then there exists \(k\) such that \(V = W_k\).
\end{theorem}

\section{Span}

\begin{definition}
    Let \(S\) be a set of vector in a vector space \(V\). The subspace spanned by \(S\) is defined to be the intersection of all subsapces of \(V\) which contains \(S\) and is denoted by \(\vspan S\). That is, \(\vspan S = \bigcap_{S \subset W} W \). By \Cref{th:IntersectionSubspace}, \(\vspan S\) is a linear subspace. Obviously, \(\vspan S\) is the smallest subspace containing \(S\) because if there were \(S \subset K \subset \vspan S\) since by definition \(\vspan S = \cap W \subset K\), then \(K = \vspan S\).
\end{definition}

\begin{example}
    Let \(S = \{0\}\) then \(\vspan S =  \bigcap_{\{0\} \subset W} W = \{0\}\). Moreover:
    \begin{equation*}
        \vspan \emptyset = \{0\} \qquad \qquad \vspan V = V
    \end{equation*}
\end{example}

\begin{theorem}
    Let \(V\) is a vector space and \( S \neq \emptyset\)
    \begin{equation*}
        \vspan S = \left \{ c_1 \alpha_1 + \dots + c_n \alpha_n \; \middle| \; \alpha_i \in S, \; c_i \in \Field, \; n \in \natural \right \}
    \end{equation*}
\end{theorem}

\begin{proof}
    Let \(L\) be the set describe above. Clearly, \(S \subset L\) and \(L\) is a subsapce of \(V\) hence \(\vspan S \subset L\). Since \(S \subset \vspan S\) and \(\vspan S\) is closed under addition and scalar multiplication then \( c_1 \alpha_1 + \dots + c_n \alpha_n \in \vspan S\) hence \(L \subset \vspan S\).
\end{proof}